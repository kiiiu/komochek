\documentclass[17pt]{extarticle}
\usepackage[utf8]{inputenc}
\usepackage[T2A]{fontenc}
%\usepackage[russian]{babel}
\usepackage{hyperref}
\hypersetup{
  colorlinks=true,
}
\usepackage{afterpage}
\usepackage{lastpage}
\usepackage[usenames,dvipsnames,svgnames,table]{xcolor}
\usepackage[margin=3cm,top=5cm]{geometry}

\makeatletter
\let\orig@Hy@EveryPageAnchor\Hy@EveryPageAnchor
\def\Hy@EveryPageAnchor{%
    \begingroup
    \hypersetup{pdfview=Fit}%
    \orig@Hy@EveryPageAnchor
    \endgroup
}
\makeatother 

\usepackage{eso-pic,graphicx}
\usepackage{svg}
\newcommand{\bgpicture}{%
	\AddToShipoutPictureBG*{%
		\includegraphics[width=\paperwidth,height=\paperheight]{graphics/oldpaper}
	}
}

\newenvironment{questpage}{\clearpage
\bgpicture}{}
\newcommand{\questend}{%
	~
	
	\begin{figure}[h]		
		\centering
		\hyperlink{page.81}{\textbf{КОНЕЦ}}
	\end{figure}
}
%\pagenumbering{gobble}
\pagestyle{empty}
\setcounter{page}{0}
\sloppy
\begin{document}
\begin{questpage}\
	\begin{figure}[t]
		\includesvg[svgpath=graphics/,width=\textwidth]{snake_orb}
	\end{figure}
	\begin{figure}[t]
		\centering
		\includesvg[svgpath=graphics/,width=0.4\textwidth]{snake_enter}
	\end{figure}
	Недавно в парке аттракционов появился необычный игровой автомат: мерцающий шар, входом в который служит хищно раскрытая пасть огромной змеи с огненно-красными глазами. Внутри этого шара кто-то неведомый задает вопросы, и если правильно ответить на все три, то отправляешься путешествовать в жаркую тропическую страну. 
	
	Ты:	\hyperlink{page.35}{не раздумывая заходишь внутрь шара} по дорожке в виде раздвоенного змеиного языка;	\hyperlink{page.15}{<<А что я забыл в этих тропиках?..>>}
\end{questpage}
\begin{questpage}
	\begin{figure}[t]
		\centering
		\includesvg[svgpath=graphics/,width=0.4\textwidth]{snake_waterfall}
	\end{figure}
	Ловко прыгая с ветки на ветку, обезьянки выводят тебя к высокой скале, по которой струится водопад. <<За этой скалой и находится Семмео>>,~-- говорят они тебе.~-- <<Но дальше мы ни за что не пойдем!>>~-- \hyperlink{page.5}{испуганно вскрикивают они и исчезают в густой листве.}
\end{questpage}
\begin{questpage}
	Тянущаяся к тебе змеиная голова застывает, однако вторая, спящая, внезапно распахивает свои глаза, полыхнувшие багровым пламенем. Она пытается дотянуться до тебя, но змеиный хвост бессильно развивает свои кольца, и ты кубарем летишь на пол. <<Я извиняюсь, но я должен вернуть шар хозяину лавки~-- взамен того, что я у него разбил,~-- поясняешь ты.~-- Иначе он не вернет мне человеческий облик>>. 
	
	<<Тот, кто дал тебе этот шип,~-- черный колдун,~-- чуть слышно произносит королева-змея.~-- Он нарочно подстроил так, чтобы ты разбил его шар. Ему нужен волшебный шар змеиного города, который удесятерит его злое могущество>>. 
	
	\hyperlink{page.74}{Поверишь}, \hyperlink{page.78}{нет}.
\end{questpage}
\begin{questpage}
	\begin{figure}[t]
		\centering
		\includesvg[svgpath=graphics/,width=0.4\textwidth]{snake_waterfall}
	\end{figure}
	На скале ты видишь высеченный знак: змея с раскрытой пастью. <<Интересно, не у ворот ли я Семмео?..>>~-- думаешь ты вслух. <<У ворот, у ворот,~-- отзывается невозмутимый какаду на ветке.~-- Можешь даже заглянуть туда, если тебе жить надоело>>. Ты видишь свое могучее отражение в озерце под водопадом и самодовольно усмехаешься. <<Заглянуть-то я бы заглянул,~-- небрежно заявляешь ты.~-- Вот только как войти? Я по скалам лазить не умею>>.~-- <<Ну попробуй запрыгнуть на скалу вон с того поваленного дерева>>,~-- покачав головой по поводу твоей самонадеянности, предлагает попугай. \hyperlink{page.7}{Ты следуешь его совету} или \hyperlink{page.12}{решаешь обойти скалу в поисках более удобного прохода}.
\end{questpage}
\begin{questpage}
	\begin{figure}[t]
		\centering
		\includesvg[svgpath=graphics/,width=0.4\textwidth]{snake_waterfall}
	\end{figure}
	Ты летишь, куда глядят твои птичьи глаза, и вскоре видишь перед собой скалу с высеченной на ней змеей с раскрытой пастью. \hyperlink{page.18}{Ты плавно опускаешься на ее вершину и...}
\end{questpage}
\begin{questpage}
	\begin{figure}[t]
		\centering
		\includesvg[svgpath=graphics/,width=0.4\textwidth]{snake_waterfall}
	\end{figure}
	\hyperlink{page.18}{Ты вскарабкиваешься на скалу, на которой высечена голова змеи с раскрытой пастью.}	
\end{questpage}
\begin{questpage}
	\begin{figure}[t]
		\centering
		\includesvg[svgpath=graphics/,width=0.4\textwidth]{snake_waterfall}
	\end{figure}
	\hyperlink{page.3}{Ты добегаешь до скалы с водопадом и останавливаешься.}
\end{questpage}
\begin{questpage}
	Забравшись на поваленное дерево, ты делаешь мощный прыжок и \hyperlink{page.18}{оказываешься на вершине скалы.}	
\end{questpage}
\begin{questpage}
	\begin{figure}[t]
		\centering
		\includesvg[svgpath=graphics/,width=0.4\textwidth]{reception}
	\end{figure}
	Ты беспрепятственно добираешься до главного храма. Внутри у дверей за столом сидит миловидная девушка со скучающим видом. Увидев тебя, она оживляется. <<Добро пожаловать!~-- радушно приветствует она тебя.~-- Я~-- хранительница этого музея!>>
	
	-- \hyperlink{page.14}{<<И сколько стоит входной билет?>>}~-- интересуешься ты у нее; \hyperlink{page.19}{<<Честно говоря, мне нужен змеиный шар>>},~-- прямо объясняешь ты.
\end{questpage}
\begin{questpage}
	Ты обходишь скалу кругом, но каменные стены неприступны, и ты \hyperlink{page.29}{возвращаешься расчищать единственный проход.}
\end{questpage}
\begin{questpage}
	\begin{figure}[t]
		\centering
		\includesvg[svgpath=graphics/,width=0.4\textwidth]{twohead_snake}
	\end{figure}
	
	Внезапно клубок змей, обвивавших тебя, разматывается, и ты обнаруживаешь себя в центре просторной каменной залы перед огромной двухголовой змеей с поблескивающими диадемами на головах. Одна голова, похоже, спит, а холодные глаза второй пристально смотрят на тебя. 
	
	<<Мы рады каждому, кто приходит за мудростью,~-- сурово произносит змея.~-- Но ты, как и все, явился сюда, чтобы обрести черное колдовское могущество!>>~-- <<Да я вообще здесь ни при чем!~-- пытаешься оправдаться ты.~-- Я только хочу вернуть свой человеческий облик>>~-- <<Ты лжешь!>>~-- шипит змея и, захлестнув тебя хвостом, притягивает к себе. У твоего лица блеснули два острых змеиных клыка... и ты вдруг вспоминаешь про шип, данный тебе торговцем. С трудом ты достаешь его... в какую голову вонзишь: \hyperlink{page.2}{в спящую} или \hyperlink{page.70}{бодрствующую}?
\end{questpage}
\begin{questpage}
	\begin{figure}[t]
		\centering
		\includesvg[svgpath=graphics/,width=0.4\textwidth]{parrot}
	\end{figure}
	
	<<Купите попугая!>>~-- звонко голосит мальчишка, бегая по рынку и размахивая клеткой. <<Слушай, ты мог бы так сильно меня не трясти, а то меня уже тошнит!>>~-- мрачно произносишь ты. Мальчишка вскрикивает и роняет клетку. \hyperlink{page.17}{От удара она раскрывается, и ты вылетаешь из нее, ворча на досадную задержку.}
\end{questpage}
\begin{questpage}
	Ты идешь вокруг скалы, но она кажется неприступной. И вдруг в одном месте ты видишь узкую щель в стене. \hyperlink{page.23}{Пробуешь протиснуться через нее} или \hyperlink{page.7}{вернешься и попытаешься допрыгнуть до скалы} со знаком змеи с поваленного дерева?
\end{questpage}
\begin{questpage}
	Крадучись ты подбегаешь (подлетаешь~-- выбери сам, что больше подходит твоей наружной оболочке: если ты слон, то тебе лучше не ползти и тем более не лететь~-- это будет довольно странно выглядеть) к одному из дворцов, окружающих храм, и осторожно заглядываешь внутрь. Никого. Твои уши улавливают слабый шелест, и вновь наступает тишина. \hyperlink{page.8}{Направишься к храму} или \hyperlink{page.71}{останешься сидеть (лежать, стоять) в засаде?}
\end{questpage}
\begin{questpage}
	\begin{figure}[t]
		\centering
		\includesvg[svgpath=graphics/,width=0.4\textwidth]{reception}
	\end{figure}
	<<Для хорошего человека~-- нисколько>>,~-- отвечает хранительница музея. Ты с недоумением себя осматриваешь. \hyperlink{page.30}{<<Внешний вид зачастую обманчив,~-- улыбается та.~-- Важно то, что внутри>>.}
\end{questpage}
\begin{questpage}
	Конечно, ведь тебе и дома так весело... \questend
\end{questpage}
\begin{questpage}
	Ты задумчиво стоишь у водопада, струящегося со скалы с высеченным на ней изображением змеи с раскрытой пастью. О том, чтобы с твоей комплекцией взбираться наверх по камням, не может быть и речи. \hyperlink{page.22}{Пробуешь проломить в скале проход} или \hyperlink{page.27}{ищешь обходной путь.}
\end{questpage}
\begin{questpage}
	На закате ты вновь прилетаешь к реке. Ты пытаешься обратиться с вопросом о древнем городе к местным пташкам, порхающим среди листвы, но те заняты своими делами. В воде у берега дремлет большой крокодил. Ты садишься на ближайшее к нему дерево. Крокодил молча открывает правый глаз. <<Не знаете ли вы, где находится город Семмео?>>~-- вежливо осведомляешься ты у него. Крокодил открывает левый глаз. <<Я не слышу тебя>>,~-- произносит он. Ты спускаешься на ветку ниже и повторяешь свой вопрос. <<Не слышу>>,~-- отвечает тот. \hyperlink{page.28}{Ты опускаешься еще ниже} или \hyperlink{page.4}{советуешь крокодилу приобрести слуховой аппарат и улетаешь.}
\end{questpage}
\begin{questpage}
	Твоим глазам открывается древний город Семмео: множество небольших каменных дворцов, которые окружают высокий величественный храм. Кругом не видно ни души. Ты облегченно вздыхаешь и \hyperlink{page.8}{напрямик устремляешься к главному храму} или \hyperlink{page.13}{предпочитаешь немного осмотреться.}
\end{questpage}
\begin{questpage}
	<<Экспонаты из музея не дарятся, не продаются и не обмениваются>>,~-- отчеканивает хранительница музея. \hyperlink{page.24}{Не споришь больше,} надеясь стащить то, что позарез тебе необходимо, или заявляешь: \hyperlink{page.25}{<<Тогда мне придется взять змеиный шар силой>>.}
\end{questpage}
\begin{questpage}
	\begin{figure}[t]
		\centering
		\includesvg[svgpath=graphics/,width=0.4\textwidth]{snake_girl}
	\end{figure}
	Миловидная хранительница музея дует на тебя, и мощный порыв ветра припечатывает тебя к стенке. Ну и что, что ты слон,~-- тебя сдует немножко позже. <<Ты пришел сюда со злыми намерениями!~-- сверкая глазами, восклицает хранительница музея.~-- За это ты будешь наказан!>> И вдруг изо всех дверей и окон извиваясь выползают тысячи больших и маленьких змей, которые обвивают тебя с ног до головы и \hyperlink{page.10}{стремительно волокут куда-то.}
\end{questpage}
\begin{questpage}
	\begin{figure}[t]
		\centering
		\includesvg[svgpath=graphics/,width=0.4\textwidth]{boar}
	\end{figure}
	Разъяренный кабан кидается на тебя, но ты отскакиваешь в сторону, и он, промчавшись мимо, со всего размаха врезается в дерево. <<Ты чего такой бешеный?~-- спрашиваешь ты его, пока он с трудом пытается сосчитать сверкающие звездочки в глазах.~-- Я же только спросить хотел...>>. 
	Робко подходят кабанята. <<Да, папа, он нас не обижал>>,~-- хором говорят они. <<Чего же вы тогда верещите как резаные?>>~-- ворчливо спрашивает кабан, поднимаясь с земли и отряхиваясь. <<А о чем ты спросить-то хотел?>>~-- обращается он к тебе. <<Мне нужно в город Семмео>>, отвечаешь ты. <<Нашел куда ходить!~-- искренне удивляется кабан.~-- Там тебя не спасут ни клыки, ни когти. \hyperlink{page.26}{Ну что ж, пойдем раз тебе надо>>.}
\end{questpage}
\begin{questpage}
	Больше ты ни до чего не додумался?.. Изображая своей головой таран, ты набиваешь на лбу здоровенную шишку, \hyperlink{page.27}{зато теперь ты отправляешься на поиски другого входа в древний город.}
\end{questpage}
\begin{questpage}
	Ты протискиваешься в скалистое отверстие и чувствуешь, что застрял. Будешь теперь здесь сидеть, пока кто-нибудь тебя не вытащит. Только к стене вдоль змеиного города и близко никто не подходит... Впрочем, отощав за два дня, \hyperlink{page.7}{ты сам выбираешься из расщелины и возвращаешься к воротам.}
\end{questpage}
\begin{questpage}
	\begin{figure}[t]
		\centering
		\includesvg[svgpath=graphics/,width=0.4\textwidth]{rounded_snake}
	\end{figure}
	Ты неторопливо идешь (летишь, топаешь) вдоль длинных рядов дворцовых экспонатов. Возможно, в другое время некоторые из них тебя бы и заинтересовали, но сейчас тебе очень хочется вернуть себе не очень важную для ее внутреннего содержания, но все же дорогую тебе наружную оболочку. Ты выходишь в просторную залу, в центре которой в круглой чаше плавает хрустальный шар со знакомой тебе зеленоватой светящейся змейков внутри. Не долго думая ты подгребаешь шар к себе. <<Этого нельзя делать>>,~-- предупреждает тебя хранительница музея, внезапно появившись перед тобою. <<Извините, но мне очень нужен этот шар>>,~-- вежливо говоришь ты,  \hyperlink{page.20}{но как только ты протягиваешь руку...}
\end{questpage}
\begin{questpage}
	\hyperlink{page.20}{Едва ты делаешь первый шаг...}
\end{questpage}
\begin{questpage}
	Кабан доводит тебя до скалы с водопадом и спешит удалиться. \hyperlink{page.3}{<<Дальше я не пойду!>>~-- кричит он тебе, убегая.}
\end{questpage}
\begin{questpage}
	Ты идешь вдоль скалы и замечаешь узкое отверстие, заваленное глыбами камней. \hyperlink{page.29}{Будешь расширять отверстие} или \hyperlink{page.9}{пойдешь дальше}?
\end{questpage}
\begin{questpage}
	\begin{figure}[t]
		\centering
		\includesvg[svgpath=graphics/,width=0.4\textwidth]{crocodile}
	\end{figure}
	<<Не зна...>>~-- начинаешь ты, и тут крокодил раскрывает свою ужасную пасть... и на этом можно было бы и закончить, \hyperlink{page.4}{но ты чудом уворачиваешься, и в крокодильих зубах остается только твой шикарный хвост.}
\end{questpage}
\begin{questpage}
	Поначалу твой хобот с непривычки соскальзывает с каменных глыб и дело продвигается медленно, однако постепенно проход становится шире, \hyperlink{page.18}{и вскоре ты расчищаешь себе путь.}
\end{questpage}
\begin{questpage}
	<<Я могу пройти посмотреть на экспонаты?>>~-- спрашиваешь ты. \hyperlink{page.24}{<<Конечно!>>~-- отвечает хранительница музея.}
\end{questpage}
\begin{questpage}
	\pagecolor[rgb]{0.1,0.1,0.1}
	\color[rgb]{1,1,1}
	<<Остроумно,~-- покачиваются перед тобой огненные буквы,~-- но ты не наблюдателен>>. Действительно кухонный чад не пишется с большой буквы... <<Я разрешаю тебе ответить на другой вопрос. На каком материке самые большие запасы подземных вод?>> \hyperlink{page.36}{В Австралии}, \hyperlink{page.41}{в Антарктиде}, \hyperlink{page.46}{<<Да кто их знает...>>.}
	\afterpage{\nopagecolor}
\end{questpage}
\begin{questpage}
	\hyperlink{page.40}{<<Какой еще зимой?!>>}
\end{questpage}
\begin{questpage}
	<<Ступай же!>>~-- машет тебе торговец. \hyperlink{page.43}{Ты печально трубишь в ответ и, понуро опустив свой хобот, медленно трогаешься в путь.}
\end{questpage}
\begin{questpage}
	\begin{figure}[t]
		\centering
		\includesvg[svgpath=graphics/,width=0.4\textwidth]{monkey}
	\end{figure}
	Теперь ты, конечно, не красавец, зато можешь проворно лазить по деревьям. \hyperlink{page.49}{А бананы здесь, надо заметить, очень даже вкусные... ням-ням...}
\end{questpage}
\begin{questpage}
	\pagecolor[rgb]{0.1,0.1,0.1}
	\color[rgb]{1,1,1}
	Змеиные челюсти смыкаются за тобой. Ты стоишь в кромешной тьме, и вдруг перед твоими глазами вспыхивает надпись: <<Вопрос первый: что такое кайман?>> Ты смуно припоминаешь, что это: \hyperlink{page.52}{кто-то вроде волка}, \hyperlink{page.45}{что-то вроде кармана с каймой,} \hyperlink{page.50}{кто-то вроде крокодила.}
	\afterpage{\nopagecolor}
\end{questpage}
\begin{questpage}
	\pagecolor[rgb]{0.1,0.1,0.1}
	\color[rgb]{1,1,1}
	\hyperlink{page.51}{<<Верно>>,~-- мерцает ответ.}
	\afterpage{\nopagecolor}
\end{questpage}
\begin{questpage}
	\pagecolor[rgb]{0.1,0.1,0.1}
	\color[rgb]{1,1,1}
	<<Ты справился с заданием!~-- вспыхивают буквы на куполе.~-- \hyperlink{page.42}{Приятного путешествия!>>.}
	\afterpage{\nopagecolor}
\end{questpage}
\begin{questpage}
	Взглянув на свою мощную лапу, ты с сомнением посматриваешь на хозяина лавки. <<И не вздумай даже!~-- предупреждает он тебя.~-- Тронешь меня~-- и на всю жизнь останешься косматой кошкой. \hyperlink{page.44}{Ступай!>>}
\end{questpage}
\begin{questpage}
	Устав проламываться через колючие заросли, ты подходишь к реке и, зачерпнув хоботом воды, устало обрушиваешь ее себе на спину. <<Поосторожнее можно?>>~-- слышишь ты недовольный голос, и на землю шлепается водяная змейка, которую ты ненароком зачерпнул вместе с водой. <<Извините,~-- говоришь ты,~-- Я нечаянно>>.~-- <<Ладно,~-- смягчается змейка.~-- Я слышала, что ты ищешь Семмео...>>~-- <<Да!~-- оживляешься ты.~-- А ты знаешь, где это?!>>~-- <<Знаю,~-- деловито отвечает та.~-- Только там тебе не поздоровится. В этом городе живут такие змеи, которые прокусят даже твою толстую кожу. И не только змеи...>>~-- <<Ничего не поделаешь, мне очень туда надо>>,~-- говоришь ты. <<Тогда иди вперед до впадения в реку ручья,~-- объясняет тебе змейка,~-- а потом пойдешь вдоль него до скалы с водопадом~-- \hyperlink{page.16}{за ней-то и находится древний город>>.}
\end{questpage}
\begin{questpage}
	\begin{figure}[t]
		\centering
		\includesvg[svgpath=graphics/,width=0.4\textwidth]{angry_bargainer}
	\end{figure}
	Приглядевшись, ты видишь, что стены и потолок увешаны и уставлены всевозможными диковинными, в основном стеклянными, штуковинами. И вдруг из-под прилавка прямо перед тобой выныривает хозяин лавки. От неожиданности ты отшатываешься и задеваешь рукой небольшой хрустальный шар со светящейся змейков внутри, \hyperlink{page.58}{который падает на пол и разбивается вдребезги.}
\end{questpage}
\begin{questpage}
	\pagecolor[rgb]{0.1,0.1,0.1}
	\color[rgb]{1,1,1}
	\hyperlink{page.52}{Ну если только в виде льда...}
	\afterpage{\nopagecolor}
\end{questpage}
\begin{questpage}
	Вчера ты еще об этом и не мечтал, а уже сегодня ты под знойным солнцем среди пальм и пестрых авочек с торговцами-зазывалами... Одной из таких лавочек ты заинтересовываешься и, войдя внутрь, в полутьме натыкаешься на что-то перед собою. Раздается жалобный стеклянный перезвон. Осторожнее, иначе ты все здесь перебьешь! А вообще мама про тебя говорит, что обычно ты: \hyperlink{page.47}{тише воды, ниже травы}, \hyperlink{page.40}{слон в посудной лавке}, \hyperlink{page.57}{коза (козел) в огороде}.
\end{questpage}
\begin{questpage}
	К вечеру ты приходишь к реке, где два слона таскают бревна. <<Вы не знаете, где находится город Семмео?>>~-- справшиваешь ты у них. <<Какой у тебя странный выговор,~-- замечает один из слонов.~-- Иностранец, что ли?..>>~-- <<Ага>>,~-- вздыхаешь ты. \hyperlink{page.39}{<<Вниз по реке,~-- указывает тот хоботом.~-- Только там деревья разрослись~-- не проломиться>>}~-- и вновь принимается за свою работу.
\end{questpage}
\begin{questpage}
	\begin{figure}[t]
		\centering
		\includesvg[svgpath=graphics/,width=0.4\textwidth]{boar}
	\end{figure}
	Ты бежишь вперед по дороге и вдруг замечаешь двух играющих под деревом кабанят. Ты подходишь к ним, но они так увлечны игрой, что не замечают тебя. <<Извините, вы не ска...>>~-- начинаешь ты, и тут кабанята с пронзительным визгом бросаются в стороны, а позади тебя раздается грозный храп. Огромный кабан с литыми кровью глазами кидается на тебя, и \hyperlink{page.6}{ты пускаешься наутек} или \hyperlink{page.21}{принимаешь угрожающе-царственную позу, встречая противника.}
\end{questpage}
\begin{questpage}
	\pagecolor[rgb]{0.1,0.1,0.1}
	\color[rgb]{1,1,1}
	\hyperlink{page.52}{Очень может быть.}
	\afterpage{\nopagecolor}
\end{questpage}
\begin{questpage}
	\pagecolor[rgb]{0.1,0.1,0.1}
	\color[rgb]{1,1,1}
	Не думай вслух~-- \hyperlink{page.52}{это было принято за ответ.}
	\afterpage{\nopagecolor}
\end{questpage}
\begin{questpage}
	Ты тихонечко рассматриваешь расставленные на полках стеклянные фигурки, шкатулки и вазочки. <<Вас что-нибудь интересует?>>~-- осведомляется хозяин лавки, появившийся будто бы ниоткуда. <<Где-то я уже это видел>>,~-- задумчиво указываешь ты рукой на хрустальный шар со светящейся змейкой внутри. В ее хищно раскрытой пасти сверкают два острых клыка, а маленькие рубиновые глазки горят багровым пламенем. <<Можно посмотреть?>>~-- спрашиваешь ты у хозяина лавки. Тот медленно утвердительно кивает, пристально глядя на тебя. \hyperlink{page.53}{Ты осторожно берешь шар в руки...}
\end{questpage}
\begin{questpage}
	\hyperlink{page.54}{<<И какого черта меня понесло в эти тропики?..>>~-- в сердцах думаешь ты, взмахивая крыльями.}
\end{questpage}
\begin{questpage}
	\begin{figure}[t]
		\centering
		\includesvg[svgpath=graphics/,width=0.4\textwidth]{tiger}
	\end{figure}
	Ты неторопливо закусываешь бананом, сидя среди широких пальмовых листьев, как вдруг замечаешь крадущегося внизу тигра. Недолго думая ты запускаешь в него банановой кожурой и пронзительно кричишь: <<Эй, ты не знаешь, где находится город Семмео?!>> Тигр от неожиданности подпрыгивает, и антилопа, на которую он охотился, тут же срывается с места и убегает. Тигр недружелюбно на тебя взглядывает. Ты извиняясь пожимаешь плечами. <<Спускайся ко мне, я покажу>>,~-- отвечает тигр. <<Я лучше здесь побуду,~-- говоришь ты, покрепче ухватываясь за ветку.~-- Мне кажется, ты не уверен, что знаешь туда дорогу>>. С соседних веток раздается хохот, и из листьев показываются мордочки обезьян. <<Здорово ты его!~-- хихикает маленькая обезьянка, зависая на ветке вниз головой.~-- \hyperlink{page.1}{Мы проводим тебя в Семмео!>>.}
\end{questpage}
\begin{questpage}
	\pagecolor[rgb]{0.1,0.1,0.1}
	\color[rgb]{1,1,1}
	<<Верно,~-- мерцают буквы, и вспыхивает следующий вопрос:~-- Где находится Чад?>> \hyperlink{page.55}{В Америке}, \hyperlink{page.59}{в Африке}, \hyperlink{page.31}{на кухне, когда гренки сгорят.}
	\afterpage{\nopagecolor}
\end{questpage}
\begin{questpage}
	\pagecolor[rgb]{0.1,0.1,0.1}
	\color[rgb]{1,1,1}
	<<И последий вопрос: когда в пустыне бывает холодно?>>~-- \hyperlink{page.75}{<<Мне бы там точно не было холодно>>,~-- пожимаешь ты плечами}; \hyperlink{page.32}{<<Зимой, наверное>>}, \hyperlink{page.37}{ночью.}
	\afterpage{\nopagecolor}
\end{questpage}
\begin{questpage}
	Сообщив загадочному вопрошателю свой ответ, ты с разочарованием видишь перед собой мерцающие слова: <<Неверный ответ>>, и змеиная пасть молча распахивается. \questend
\end{questpage}
\begin{questpage}
	Шар выскальзывает из твоих рук и, упав на пол, разбивается вдребезги. \hyperlink{page.58}{Змейка тускнеет, и ее огненные глазки потухают.}
\end{questpage}
\begin{questpage}
	\begin{figure}[t]
		\centering
		\includesvg[svgpath=graphics/,width=0.4\textwidth]{fishnet}
	\end{figure}
	<<Интересно, и куда это я лечу?~-- думаешь ты, когда под тобой зашелестела река.~-- Надо хотя бы напиться>>. Ты спускаешься к реке и вдруг оказываешься в ловко наброшенной сети. Ты пытаешься вырваться из нее, но прочная сеть все сильнее запутывает тебя, и ты затихаешь. \hyperlink{page.11}{Довольный мальчишка распутывает сеть, запихивает тебя в клетку и уходит от реки.}
\end{questpage}
\begin{questpage}
	\hyperlink{page.52}{Учи географию.}
\end{questpage}
\begin{questpage}
	\begin{figure}[t]
		\centering
		\includesvg[svgpath=graphics/,width=0.4\textwidth]{surprised_bargainer}
	\end{figure}
	<<Я принес вам то, что был должен,~-- говоришь ты торговцу, протягивая ему шар.~-- Превращайте меня обратно>>. Торговец выхватывает у тебя шар и собирается уходить. <<Вы же обещали!>>~-- возмущаешься ты. <<Мне некогда говорить с тобой>>,~-- пренебрежительно заявляет тот. И вдруг зеленоватая змейка внутри шара мерцает ярче, и шар начинает увеличиваться. <<Это еще что такое?>>~-- недоуменно бормочет торговец. \hyperlink{page.77}{И тут шар...}
\end{questpage}
\begin{questpage}
	Ты с любопытством рассматриваешь разные стеклянные вещицы, расставленные на полках и свисающие с потолка, как вдруг твое внимание привлекает небольшой хрустальный шар со светящейся змеей внутри, хищно раскрывшей свою пасть. \hyperlink{page.53}{<<О, прямо как в парке!>>~-- удивляешься ты, хватая шар.}
\end{questpage}
\begin{questpage}
	\begin{figure}[t]
		\centering
		\includesvg[svgpath=graphics/,width=0.4\textwidth]{angry_bargainer}
	\end{figure}
	<<Что ты наделал?!~-- гневно восклицает хозяин лавки.~-- Этот шар был моим талисманом! Человек, не охраняемый мудростью змеиного шара, лишается милости богов! Теперь ты должен принести мне такой же>>.~-- <<Где ж я возьму-то?..>>~-- растерянно лепечешь ты. <<В древнем городе Семмео>>,~-- отвечает торговец. <<И где находится этот... Семь ё?..>>~-- удрученно интересуешься ты. <<Семмео~-- змеиное жало,~-- уточняет хозяин.~-- Этот город находится в глубине джунглей, и тебе придется самому отыскать туда дорогу. Возьми с собой этот шип~-- он тебе пригодится. А чтобы ты не вздумал сбежать от меня, я превращу тебя в какое-нибудь животное...>> Ну что ж, выбирай: \hyperlink{page.33}{в слона}, \hyperlink{page.34}{обезьяну}, \hyperlink{page.38}{во льва} или \hyperlink{page.48}{попугая.}
\end{questpage}
\begin{questpage}
	\pagecolor[rgb]{0.1,0.1,0.1}
	\color[rgb]{1,1,1}
	\hyperlink{page.51}{<<Правильно>>,~-- вспыхивают буквы.}
	\afterpage{\nopagecolor}
\end{questpage}
\begin{questpage}
	\begin{figure}[t]
		\centering
		\includesvg[svgpath=graphics/,width=0.35\textwidth]{rabbit}
	\end{figure}
	Сначала ты стараешься идти вдоль берега, но потом плотная стена колючих зарослей вынуждает тебя отступить вглубь леса. Ты бредешь вперед наугад, пока не понимаешь, что заблудился. Вдруг ты слышишь совсем близко чей-то голос. Обрадованный, то где-то рядом человек, который может помочь тебе, ты устремляешься к нему и обнаруживаешь, что голос доносится... из-под земли. Ты вовремя заметил, что стоишь на краю ямы-ловушки, полуприкрытой пальмовыми листьями. 
	
	Ты заглядываешь в нее и видишь упитанного грустного кролика, сидящего рядом с огромным пучком листьев. <<Неужели я никогда не выберусь отсюда!~-- горько сетует он.~-- Остались без кормильца мои детки!>> \hyperlink{page.65}{Пройдешь мимо} или \hyperlink{page.61}{попробуешь помочь кролику?}
\end{questpage}
\begin{questpage}
	\begin{figure}[t]
		\centering
		\includesvg[svgpath=graphics/,width=0.35\textwidth]{rabbit}
	\end{figure}
	<<Эй!>>~-- окликаешь ты его. Кролик округляет глаза. <<Ничего себе!~-- восклицает он.~-- Ты понимаешь язык животных?! Ах, ну да,~-- спохватывается он,~-- ты же владелец змеиного шара!>>~-- <<Я иду вернуть его королеве змей,~-- уточняешь ты.~-- Ты не мог бы показать мне дорогу в древний город?>>~-- <<Конечно!~-- уверяет тебя кролик.~-- Ты только вытащи меня отсюда!>> Ты поднимаешь с земли длинную ветку и опускаешь ее в яму. Кролик хватается одной лапой за ветку, а другой прижимает к себе увесистый пучок листьев. Ты тянешь ветку на себя, но у тебя ничего не получается. <<А ты бы оставил свою зелень в яме, а?>>~-- советуешь ты ему. <<Не могу!~-- отвечает кролик.~-- Я несу ее своим детям>>.~-- <<Тогда я не смогу вытащить тебя!>>~-- объясняешь ты ему. Кролик вздыхает и с надеждой смотрит на тебя, прижимая к себе листья. <<Ну и сиди тогда здесь!>>~-- \hyperlink{page.65}{говоришь ты, уходя,} или \hyperlink{page.69}{делаешь еще одну попытку вытащить кролика.}
\end{questpage}
\begin{questpage}
	Кролик с воплем: <<Помогите! Человек!>>~-- хватая свою охапку листьев и вжимаясь в земляную стену, испуганно смотрит на тебя. <<Ну и человек>>,~-- мрачно произносишь ты, утешаясь тем, что хотя бы змеиный шар не разбился после твоего полета. Кролик пугается еще больше. <<Ты понимаешь язык животных?!~-- потрясенно восклицает он, а потом, вглядевшись в тебя, понимающе качает головой~-- Ты ведь владеешь змеиным шаром!>>~-- <<Я иду вернуть его королеве,~-- уточняешь ты.~-- Не знаешь, как добраться до древнего города?>>~-- \hyperlink{page.67}{<<Это я знаю!~-- уверенно отвечает кролик.~-- Я не знаю, как выбраться из этой ямы!>>.}
\end{questpage}
\begin{questpage}
	\begin{figure}[t]
		\centering
		\includesvg[svgpath=graphics/,width=0.4\textwidth]{smiling_girl}
	\end{figure}
	В главном храме древнего города тебя радушно встречает уже знакомая тебе хранительница музея. <<Я тут принес ваш шар,~-- говоришь ты, протягивая ей хрустальную драгоценность.~-- Передайте его королеве>>.~-- <<У тебя мудрая душа,~-- улыбается та.~-- А мудрым всегда сопутствует удача>>. Ты прощаешься с ней и выходишь. Когда у ворот ты оборачиваешься, то замечаешь, что \hyperlink{page.80}{в дверях храма стоит королева-змея с диадемами на головах...}
\end{questpage}
\begin{questpage}
	\begin{figure}[t]
		\centering
		\includesvg[svgpath=graphics/,width=0.35\textwidth]{talk_with_monkeys}
	\end{figure}
	\small
	Что-то не узнаешь ты здешних мест... Вдруг впереди в кустах ты слышишь чьи-то голоса: похоже, ругаются две женщины. Ты осторожно раздвигаешь лианы и видишь готовых подраться мартышек, одна из которых держи в руках связку бананов. <<Я первая их увидела!>> -- кричит обделенная мартышка. <<Как бы не так -- поищи себе другие!>> -- вопит вторая. <<Ну как вам не стыдно?! -- укоризненно произносишь ты. -- Неужели так трудно поделить эти бананы?!>> Мартышки разом умолкают и с изумлением таращатся на тебя. <<ВОт так штука! -- восклицает одна. -- Этот человек понимает язык животных!>> Действительно... <<И ничего странного! -- заявляет какая-то синяя пичужка на ветке. -- У него же шар змеиного города! Он несет его обратно>>. <<Кстати, далеко еще до города?>> -- спрашиваешь ты. <<Теперь действительно далеко, -- подтверждает птичка. -- Ты ведь идешь в другую сторону>>. -- <<А что же ты мне раньше об этом не сказала?!>> -- возмущаешься ты. <<Ты и не спрашивал, - невозмутимо отвечает птичка. -- Мало ли куда еще тебе надо...>> \hyperlink{page.60}{Придется повернуть обратно.}
\end{questpage}
\begin{questpage}
	Ты поворачиваешься, чтобы уйти, но вдруг земля осыпается под твоими ногами, \hyperlink{page.62}{и ты кубарем летишь в яму.}
\end{questpage}
\begin{questpage}
	Ты снова отправляешься в джунгли, теперь уже в своем истинном облике. Вскоре перед тобой заблестела река, и ты вспоминаешь, что дальше тебе нужно: \hyperlink{page.60}{вниз по течению}, \hyperlink{page.64}{вверх по течению.}
\end{questpage}
\begin{questpage}
	\begin{figure}[t]
		\centering
		\includesvg[svgpath=graphics/,width=0.4\textwidth]{scared_rabbit}
	\end{figure}
	Внезапно стены ямы оживают, и ты видишь, что они кишат змеями! Кролик вскрикивает от ужаса, но змеи покрывают дно ямы, поднимая вас все выше, и наконец вы выбираетесь из нее. <<Спасибо!>> -- восклицает радостный кролик, собираясь убегать со своей охапкой листьев. <<Подожди! -- -останавливаешь ты его. -- Ты же обещал показать мне дорогу к змеиному городу!>> -- <<А, ну да! -- вспоминает кролик, почесывая ухо свободной лапкой. -- Змеиный город... это туда... или туда?..>> -- <<Ты что, не знаешь?>> хх подозрительно спрашиваешь ты у него. <<Ага, -- не смущаясь, отвечает тот. -- А ты вон у них спроси, -- кивает он на проворных змеек. -- Они-то уж точно знают>>. -- \hyperlink{page.63}{<<Мы проводим тебя!>> -- шипят змейки, и ты отправляешься вслед за ними.}
\end{questpage}
\begin{questpage}
	Торговец оборачивается, и шар в его руках вспыхивает ослепительным зеленым светом. Через мгновение вместо тебя на землю шлепается жирная мохнатая гусеница. Ну что ж, не самый худший вариант, только скорее убегай под листочек, а то тобой уже заинтересовалась вон та птичка на ветке! \questend
\end{questpage}
\begin{questpage}
	Ты с усилием тянешь ветку, как вдруг земля под тобой осыпается, \hyperlink{page.72}{и ты летишь в яму.}
\end{questpage}
\begin{questpage}
	\begin{figure}[t]
		\centering
		\includesvg[svgpath=graphics/,width=0.4\textwidth]{snake}
	\end{figure}
	Тянущаяся к тебе змеиная голова замирает, но глаза второй внезапно распахиваются и вспыхивают зловещим багровым пламенем... Ты снова меняешь свой облик, но теперь уже навсегда. Добро пожаловать в ползучее царство, новый подданный змеиного города! \questend
\end{questpage}
\begin{questpage}
	В засаде ты находишься довольно долго. Наступает ночь, и на город наползают тьма и сырой и пронизывающий холод, но ты упорно не двигаешься с места, несмотря на то что у тебя от холода свело лапы (скрючило крылья, отмерз хвост, уши свернулись в трубочку). Утром, стуча зубами (лязгая клыками, хлопая ушами, потрясая перьями) и так ничего и не дождавшись, \hyperlink{page.8}{ты направляешься к храму.}
\end{questpage}
\begin{questpage}
	Ладно, хотя бы змеиный шар остался цел... <<Ну и что теперь делать?>> -- укоризненно спрашиваешь ты у кролика. \hyperlink{page.67}{Тот виновато поджимает уши.}
\end{questpage}
\begin{questpage}
	<<Вот ваш шар! -- объявляешь ты торговцу, положив перед ним свою хрупкую добычу. -- А теперь сделайте из меня что было>>. -- <<Неужели?! -- алчно восклицает торговец, протягивая к шару трясущиеся руки. -- Наконец-то он у меня...>> Он осторожно поднимает шар и собирается уходить. <<Эй, а как же я?>> -- окликаешь ты его. <<Отвяжись! -- огрызается торговец. -- Мне не до тебя. Будешь липнуть ко мне -- превращу тебя в дождевого червя или во что еще похуже>>. -- <<Ну и наглеж!>> -- возмущаешься ты. \hyperlink{page.68}{Догонишь торговца и треснешь его лапой (кокосом, клювом, хоботом)} или \hyperlink{page.79}{вкрадчиво произнесешь: <<А шарик-то ненастоящий>>.}
\end{questpage}
\begin{questpage}
	<<И что же мне теперь делать?..>> -- растерянно спрашиваешь ты.. <<Я дам тебе другой шар, -- отвечает змея. -- Отдай его тому торговцу -- увидишь, что будет>>. Ты осторожно вынимаешь шип, и \hyperlink{page.56}{змея протягивает тебе шар со светящейся змейкой.}
\end{questpage}
\begin{questpage}
	\pagecolor[rgb]{0.1,0.1,0.1}
	\color[rgb]{1,1,1}
	\begin{figure}[t]
		\centering
		\includesvg[svgpath=graphics/,width=0.4\textwidth]{frosty_boy}
	\end{figure}
	\hyperlink{page.52}{Сиди дома, морозоустойчивый ты наш...}
	\afterpage{\nopagecolor}
\end{questpage}
\begin{questpage}
	Ну что ж, ты возвращаешься домой, и благодаря твоим стараниям скоро на земле будут жить одни змеи. \questend
\end{questpage}
\begin{questpage}
	\begin{figure}[t]
		\centering
		\includesvg[svgpath=graphics/,width=0.4\textwidth]{snaked_bargainer}
	\end{figure}
	...раскалывается надвое, и змейка, упав на пол, разваливается на кусочки. Каждый из этих кусочков становится живой проворной змеей, которая набрасывается на торговца, и тот в один миг превращается в такую же зеленую змейку и с шипением уползает в кусты. Половинки шара срастаются, и внутри него вновь появляется мерцающая змейка. Ты озадаченно трешь свой лоб лапой (когтем, крылом, хоотом) и замечаешь, что вместо всего этого у тебя обыкновенная рука. Как это -- пора домой? А кто шар возвращать будет?.. \hyperlink{page.76}{<<Это меня не волнует, я и так уже по джунглям нагулялся>>}; \hyperlink{page.66}{<<И то верно -- а то ведь если этот шарик каждого в змею превращать будет -- этих ползучих разведется ужас сколько!..>>.}
\end{questpage}
\begin{questpage}
	\begin{figure}[t]
		\centering
		\includesvg[svgpath=graphics/,width=0.4\textwidth]{twohead_snake_2}
	\end{figure}
	<<Даже если то, о че ты говоришь, -- правда, у меня нет выбора: если я не принесу ему шар, мне не стать снова человеком>>, -- произносишь ты, уходя. <<Он станет могущественным колдуном, он обманет тебя!>> -- кричит тебе вслед королева-змея, но \hyperlink{page.73}{ты забираешь шар и уходишь.}
\end{questpage}
\begin{questpage}
	Воспользовавшись минутным замешательством торговца-обманщика, ты бросаешься к нему и в отчаянии выбиваешь шар из его рук. \hyperlink{page.77}{Тот падает на землю, и....}
\end{questpage}
\begin{questpage}
	\begin{figure}[t]
		\centering
		\includesvg[svgpath=graphics/,width=0.4\textwidth]{lady_with_dog}
	\end{figure}
	Жаль, конечно, что теперь, когда ты вернул шар, ты перестал понимать язык животных, но это все же лучше, чем вероятность в любой момент стать змеей. Возле небольшой деревеньки ты садишься в автобус. В нем много народа, и ты, нечаянно упершись локтем во что-то мягкое, слышишь позади себя недовольный голос: <<Можно поосторожнее?>> -- <<Извините, но меня тоже толкнули>>, -- отвечаешь ты, оборачиваешься и видишь солидную даму в шляпе и с большой сумкой в руках. <<Я ничего не говорила!>> -- удивляется та. Из сумки высовывается лохматая собачья голова и сердито чихает, глядя на тебя. <<Ездят тут всякие, толкаются>>, -- ворчит собака. Да уж, пойдут ли тебе впрок твои знания -- неизвестно, но то что очередные приключения с таким прощальным подарком королевы змей тебе обеспечены, -- это точно!.. \questend
\end{questpage}
\begin{questpage}
	Сделано по материалам прекрасной газеты <<Комочек>> (<<\reflectbox{k}oMok>>). Спасибо за часы и дни, проведенные в ваших <<бродилках>>.
	\begin{figure}[h]
		\centering
		\includesvg[svgpath=graphics/,width=0.7\textwidth]{komoch}
	\end{figure}
\end{questpage}
\end{document}
